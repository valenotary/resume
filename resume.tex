% Credit to the scimisc-cv.sty template from Scimisc (on Overleaf)
\documentclass{article}
\usepackage{scimisc-cv}

\title{Resume}
\author{Franz Anthony Varela (Template courtesy of Scimisc)}
\date{May 2020}

%% These are custom commands defined in scimisc-cv.sty
\cvname{Franz Anthony Varela}
\cvpersonalinfo{
El Dorado Hills, CA \cvinfosep
916-430-2109 \cvinfosep
fvarela3@uw.edu \cvinfosep
LinkedIn: Franz Anthony Varela
}

\begin{document}

% \maketitle %% This is LaTeX's default title constructed from \title,\author,\date

\makecvtitle %% This is a custom command constructing the CV title from \cvname, \cvpersonalinfo

% \section{Summary}
% \begin{itemize}
% \item Interdisciplinary scientist with skills and experience in immunology, genomics, and molecular biology
% \item Led collaborative projects, resulting in 6 peer-reviewed publications, including 5 high impact first-authored publications, and 3 patents
% \item Deep understanding of genomic data analysis and visualization
% \item Self-motivated, problem-solving and collaborative scientist with excellent communication skills
% \item Looking to contribute to use computational methods to push forward Gene Therapy projects towards the clinic
% \end{itemize}

% \section{Research Experience}

%% Another custom command provide by scimisc-cv.sty.
%% First two argumetns are typeset on the first line in bold; 3rd and 4th arguments are typset on second line in italics. 2nd, 3rd and 4th arguments are OPTIONAL
% \cvsubsection{Amgen}[Thousand Oaks]
% [Scientist I][Sept 2018 to present]
%
% \begin{itemize}
% \item Led 3 highly collaborative projects all focused on the validation of novel therapeutic vectors in animal disease models (neurodegenerative diseases)
% \item Managed a small team of 2 technical reports
% \item Responsible for designing experiments that drove the project forward towards IND submission
% \item Oversaw the PK/PD, and toxicology studies conducted by various CROs
% \item This project led to the submission of 3 publications and 1 patent
% \end{itemize}
%
% %% An example of leaving an argument empty
% \cvsubsection{Massachusetts General Hospital}[][Post doctoral Fellow][July 2014 to Sept 2018]
%
% \begin{itemize}
% \item Led 2 primary projects focused on the developing a library of small molecules targeting pathways involved in neurodegenerative diseases
% \item Developed high-throughput screening assays with novel functional readout (target validation assays)
% \item Used computational methods to develop novel small molecules that fit target profile
% \item These projects led to the submission of 2 publications and 2 patents
% \end{itemize}


\section{Education}
\cvsubsection{University of Washington, Bothell}[Bothell, WA][M.S. in CS\&SE (\textbf{GPA:3.95})][Expected Graduation Date: June 2022]
\begin{itemize}
	\item \textbf{Relevant Coursework}: Machine Learning | High Performance Computing | Advanced Computer Graphics | Evaluating Software Design
\end{itemize}
\cvsubsection{University of California, Merced}[Merced, CA][B.S. in CS\&E (\textbf{GPA:3.82})][Aug 2017 - May 2020]
\begin{itemize}
  \item Graduated with High Honors 
  \item \textbf{Relevant Coursework}: Deep and Reinforcement Learning | Software Engineering | Numerical Methods | Database Systems | Computer Architecture | Linear Analysis | Algorithm Design and Analysis | Object-Oriented Programming
\end{itemize}
% \begin{itemize}
% \item PhD, Computational/Molecular Biology, Harvard University, 2014
% \item BS, Biology, University of Massachusetts, 2010
% \end{itemize}

\section{Teaching and Mentoring Experience }
\cvsubsection{The Math Center}[Merced, CA][Remedial Tutor][Sept 2019 - Aug 2020]
\begin{itemize}
  \item Tutor university students in a wide range of courses (e.g. Calc. I-III, Lin. Alg, etc.)
  \item Deepen their understanding of mathematical concepts by teaching the material in different ways
  \item Offer remote learning options, such as Discord or Zoom
\end{itemize}
\cvsubsection{Summer Bridge Program}[Merced, CA][Workshop Co-Host][July 2020 - Aug 2020]
\begin{itemize}
  \item Host an hour-long weekly workshop for incoming freshmen at UC Merced over Zoom (2 groups of 25 students each)
  \item Provide assistance in adjusting to college courses, advices on succeeding in their programs, and encourage them to be proactive
\end{itemize}

\section{Technical Skills}
\begin{itemize}
  \item Languages: C++, C\#, C, Python, Java, SQL, LaTeX
  \item Technologies/Environments: Linux (Ubuntu Subshell), Windows, Jupyter Notebook, Git, SciKit-Learn, TensorFlow (2.0), PyTorch, OpenGL, Blender, Processing 3, Arduino, Android SDK, Sqlite3, Unity, Aseprite, FL Studio, CUDA
\end{itemize}

\section{Research Experience}
\cvsubsection{"2-Dimensional Online Scheduling"}[Merced, CA][Undergraduate Researcher][Jul 2019 - Aug 2019]
\begin{itemize}
  \item Assisted Prof. Sungjin Im in his research on developing a new framework for analyzing low dimensional online scheduling problems, specifically in the 2-Dimensional setting
  \item Created algorithms in C++, and performed multiple analyses on the runtime of the implementation
  \item Provided insight into the challenges, limitations, and correctness of the Priority(Max) algorithm with our implementation
\end{itemize}

\section{Projects}
\cvsubsection{Multiclass AdaBoost Ensemble}[Merced, CA][Project Contributor][Oct 2019 - Dec 2019]
\begin{itemize}
  \item An implementation of the Multiclass AdaBoost decision forest in a group of 3 other students
  \item Implemented using Python and Jupyter Notebook, as well as analyzed our model's accuracy and runtime against SciKit-Learn's benchmark on benchmark datasets
  \item Achieved a 97\% test accuracy on the MNIST dataset
\end{itemize}
\cvsubsection{"Code-Dude"}[Merced, CA][Project Lead, Developer][Feb 2020]
\begin{itemize}
  \item Collaborated in a 3-person team to develop a 2D pixel-platformer
  \item Acted as the project manager for the group, implemented most of the features using C\# and Unity, and created with FL Studio and Aseprite
  \item Won Best Design at HackMerced V
\end{itemize}
\cvsubsection{NEAT with OpenAI Gym}[Merced, CA][Solo Developer][May 2020]
\begin{itemize}
  \item Implemented the \textit{NeuroEvolution of Augmenting Topologies} (NEAT) algorithm using Python and a variety of RL environments by Gym (e.g. CartPole-v1, BipedalWalker-v2, MsPacman-v0, etc.)
  \item Created the genome networks from scratch, as well as developed a unique forward propagation algorithm to support the evolution of new connections and nodes
  \item Successfully solved the CartPole-v1 environment in an average of 307.42 evaluations across its 33 trials
\end{itemize}

%\section{Leadership}
%\cvsubsection{Society of Asian Scientists and Engineers, UC Merced Chapter}[Merced][Vice President][Aug 2018 - May 2020]
%\begin{itemize}
%  \item A professional development club for university students, where we host resume workshops and offer group company tours
%  \item Handle internal affairs, planning workshops and general meetings with other board members
%  \item Hosted one of the largest engineering club networking events held on our campus
%\end{itemize}

\section{Awards}
\begin{itemize}
  \item Calvin E. Bright Engineering Scholarship
\end{itemize}

\end{document}
